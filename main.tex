\documentclass[12pt,a4paper]{report}
\usepackage[utf8]{inputenc}
\usepackage[T1]{fontenc}
\usepackage[brazilian]{babel}
\usepackage{graphicx}
\usepackage{booktabs}
\usepackage{siunitx}
\usepackage{multirow}
\usepackage{float}
\usepackage{hyperref}
\usepackage{tikz}
\usepackage{pgfplots}
\usepackage{pgf-pie}
\usepackage{pgfplotstable}
\usepackage{xcolor}
\usepackage{listings}
\usetikzlibrary{positioning,arrows.meta,calc,shapes.geometric,patterns}
\usepackage{pgfgantt}
\pgfplotsset{compat=1.18}

\tikzset{
    every picture/.style={
        line width=0.5pt,
        line cap=round,
        line join=round
    }
}

% Cores personalizadas para gráficos
\definecolor{chart1}{RGB}{70,130,180}
\definecolor{chart2}{RGB}{95,158,160}
\definecolor{chart3}{RGB}{72,61,139}

% Configurar margens da página
\usepackage[margin=2.5cm]{geometry}

\pgfplotsset{compat=1.18}

\title{MediPump Technologies\\Documentação da Instalação de Manufatura}
\author{Equipe de Documentação Técnica}
\date{\today}

\begin{document}
\maketitle
\tableofcontents

\chapter{Introdução e Visão Geral}

\section{Visão Geral da Instalação}
\subsection{Especificações da Instalação}

A instalação de manufatura da MediPump Technologies representa uma sala limpa de última geração Classe 100.000 (ISO 8) dedicada à produção de dispositivos médicos de precisão.

% Gráfico de Especificações da Instalação
\begin{figure}[H]
\centering
\begin{tikzpicture}
\begin{axis}[
    ybar,
    bar width=25pt,
    ylabel={Área (metros quadrados)},
    symbolic x coords={Espaço Total, Sala Limpa, Produção, Armazenamento, CQ},
    xtick=data,
    nodes near coords,
    nodes near coords align={vertical},
    width=\textwidth,
    height=8cm,
    title={Distribuição de Área da Instalação}
]
\addplot[fill=chart1] coordinates {
    (Espaço Total,5000)
    (Sala Limpa,2000)
    (Produção,2500)
    (Armazenamento,1100)
    (CQ,400)
};
\end{axis}
\end{tikzpicture}
\caption{Distribuição do Espaço da Instalação}
\label{fig:facility-space}
\end{figure}

\subsection{Capacidade de Produção}
\begin{figure}[H]
\centering
\begin{tikzpicture}
\begin{axis}[
    title={Metas de Produção Anual},
    ylabel={Unidades},
    symbolic x coords={X1,X1 Pro,Mini},
    xtick=data,
    nodes near coords,
    nodes near coords align={vertical},
    width=\textwidth,
    height=8cm,
    ybar,
    bar width=25pt,
]
\addplot[fill=chart2] coordinates {
    (X1,25000)
    (X1 Pro,15000)
    (Mini,10000)
};
\end{axis}
\end{tikzpicture}
\caption{Metas de Produção Anual por Modelo}
\label{fig:production-targets}
\end{figure}

\section{Descrição do Produto}
\subsection{Especificações da Linha de Produtos}

\begin{table}[H]
\centering
\caption{Comparação da Linha de Produtos}
\label{tab:product-comparison}
\begin{tabular}{@{}llll@{}}
\toprule
Especificação & MediPump X1 & MediPump X1 Pro & MediPump Mini \\
\midrule
Dimensões & 8,5×5,3×2,1 cm & 8,0×5,0×1,9 cm & 6,5×4,0×1,8 cm \\
Peso & 75g & 65g & 45g \\
Duração da Bateria & 7 dias & 10 dias & 5 dias \\
Reservatório & 3,0 mL & 3,5 mL & 2,0 mL \\
Preço Alvo & R\$ 19.000 & R\$ 22.500 & R\$ 16.000 \\
\bottomrule
\end{tabular}
\end{table}

% Gráfico de Recursos dos Produtos
\begin{figure}[H]
\centering
\begin{tikzpicture}
    % Definir os eixos
    \foreach \angle [count=\xi] in {0, 72, 144, 216, 288}
        \draw[gray] (0,0) -- (\angle:5);
    
    % Grade circular
    \foreach \r in {1,2,3,4,5}
    {
        \draw[gray, thin] (0,0) circle (\r);
    }
    
    % Plotagem para X1
    \draw[chart1, thick] plot coordinates {
        (0:3.5)    % Duração da Bateria
        (72:3.75)  % Eficiência de Tamanho
        (144:4)    % Recursos
        (216:4.25) % Desempenho de Preço
        (288:4.5)  % Interface do Usuário
    } -- cycle;
    
    % Plotagem para X1 Pro
    \draw[chart2, thick] plot coordinates {
        (0:5)      % Duração da Bateria
        (72:4.25)  % Eficiência de Tamanho
        (144:5)    % Recursos
        (216:3.75) % Desempenho de Preço
        (288:4.75) % Interface do Usuário
    } -- cycle;
    
    % Plotagem para Mini
    \draw[chart3, thick] plot coordinates {
        (0:2.5)    % Duração da Bateria
        (72:5)     % Eficiência de Tamanho
        (144:3)    % Recursos
        (216:4.5)  % Desempenho de Preço
        (288:4.25) % Interface do Usuário
    } -- cycle;
    
    % Rótulos
    \node at (0:5.5) {Duração da Bateria};
    \node at (72:5.5) {Eficiência de Tamanho};
    \node at (144:5.5) {Recursos};
    \node at (216:5.5) {Desempenho de Preço};
    \node at (288:5.5) {Interface do Usuário};
    
    % Legenda
    \node[right] at (6,2) {\textcolor{chart1}{--- X1}};
    \node[right] at (6,1) {\textcolor{chart2}{--- X1 Pro}};
    \node[right] at (6,0) {\textcolor{chart3}{--- Mini}};
\end{tikzpicture}
\caption{Comparação de Recursos dos Produtos}
\label{fig:feature-comparison}
\end{figure}

\section{Fluxo do Processo de Manufatura}
\begin{figure}[H]
\centering
\begin{tikzpicture}[
    auto,
    block/.style={
        rectangle,
        draw=black,
        thick,
        fill=white,
        text width=2.5cm,
        align=center,
        rounded corners,
        minimum height=1cm
    },
    line/.style={draw, thick, -latex'},
]
   % Primeira linha
   \node[block] (raw) {Matérias-Primas};
   \node[block, right=1.5cm of raw] (molding) {Moldagem por Injeção};
   \node[block, right=1.5cm of molding] (qc1) {CQ de Componentes};

   % Segunda linha
   \node[block, below=2cm of molding] (assembly) {Montagem};
   \node[block, right=1.5cm of assembly] (testing) {Testes};

   % Terceira linha
   \node[block, below=2cm of assembly] (sterilization) {Esterilização};
   \node[block, right=1.5cm of sterilization] (packaging) {Embalagem};
   \node[block, right=1.5cm of packaging] (finalqc) {CQ Final};

   % Conectar blocos com setas
   \path [line] (raw) -- (molding);
   \path [line] (molding) -- (qc1);
   \path [line] (qc1) -- (assembly);
   \path [line] (assembly) -- (testing);
   \path [line] (testing) -- node[right] {Aprovado} (packaging);
   \path [line] (testing) -- node[left] {Reprovado} (assembly);
   \path [line] (packaging) -- (finalqc);
   
   % Adicionar etapa de esterilização
   \path [line] (testing) -- (sterilization);
   \path [line] (sterilization) -- (packaging);

   % Loops de feedback de qualidade
   \draw [line] (finalqc) -- ++(1,0) -- ++(0,4) -- ++(-8,0) -- (assembly);
   
   % Rótulos para estágios do processo
   \node[above=0.2cm of raw] {\textbf{Estágio 1}};
   \node[above=0.2cm of qc1] {\textbf{Estágio 2}};
   \node[left=0.2cm of assembly] {\textbf{Estágio 3}};
   \node[left=0.2cm of sterilization] {\textbf{Estágio 4}};

\end{tikzpicture}
\caption{Fluxo do Processo de Manufatura}
\label{fig:process-flow}
\end{figure}

\chapter{Características do Processo}

\section{Características da Instalação}
\subsection{Requisitos Físicos}

\begin{table}[H]
\centering
\caption{Requisitos Físicos da Instalação}
\label{tab:physical-req}
\begin{tabular}{@{}ll@{}}
\toprule
Requisito & Especificação \\
\midrule
Capacidade de Carga do Piso & \SI{1000}{\kilo\gram\per\square\meter} \\
Altura do Teto & \SI{4,5}{\meter} \\
Controle de Temperatura & \SI{20}{\celsius} ±\SI{0,5}{\celsius} \\
Controle de Umidade & 45\% ±5\% UR \\
\bottomrule
\end{tabular}
\end{table}

% Gráfico de Distribuição de Equipamentos
\begin{figure}[H]
\centering
\begin{tikzpicture}
\begin{axis}[
    title={Distribuição de Equipamentos por Tipo},
    ylabel={Número de Unidades},
    symbolic x coords={Moldagem por Injeção,Montagem,Testes,Embalagem,CQ},
    xtick=data,
    nodes near coords,
    nodes near coords align={vertical},
    width=\textwidth,
    height=8cm,
    ybar,
    bar width=25pt,
]
\addplot[fill=chart3] coordinates {
    (Moldagem por Injeção,5)
    (Montagem,4)
    (Testes,8)
    (Embalagem,2)
    (CQ,6)
};
\end{axis}
\end{tikzpicture}
\caption{Distribuição de Equipamentos}
\label{fig:equipment-dist}
\end{figure}

\section{Análise do Processo}
% Gráfico de Eficiência do Processo
\begin{figure}[H]
\centering
\begin{tikzpicture}
\begin{axis}[
    title={Métricas de Eficiência do Processo},
    ylabel={Porcentagem (\%)},
    symbolic x coords={Balanceamento de Linha,Utilização,Taxa de Qualidade,OEE},
    xtick=data,
    nodes near coords,
    nodes near coords align={vertical},
    width=\textwidth,
    height=8cm,
    ybar,
    bar width=25pt,
]
\addplot[fill=chart1] coordinates {
    (Balanceamento de Linha,85)
    (Utilização,90)
    (Taxa de Qualidade,98)
    (OEE,82)
};
\end{axis}
\end{tikzpicture}
\caption{Métricas de Eficiência do Processo}
\label{fig:process-efficiency}
\end{figure}

\section{Design do Layout}

\begin{figure}[H]
    \centering
    \includegraphics[width=0.8\textwidth]{layout.pdf}  % 80% da largura
    \caption{Layout da Instalação (Versão Reduzida)}
    \label{fig:factory-layout-small}
\end{figure}

% Gráfico de Pizza de Alocação de Área
\begin{figure}[H]
\centering
\begin{tikzpicture}
\pie[radius=4]{
    25/Produção,
    20/Armazenamento,
    15/Montagem,
    15/Testes,
    15/Embalagem,
    10/CQ
}
\end{tikzpicture}
\caption{Alocação de Área da Instalação}
\label{fig:area-allocation}
\end{figure}


\chapter{Documentação do Processo}

\section{Análise da Rota de Manufatura}

\subsection{Análise do Tempo de Processo}
\begin{figure}[H]
\centering
\begin{tikzpicture}
\begin{axis}[
    title={Análise de Duração das Etapas do Processo},
    ylabel={Tempo (segundos)},
    xlabel={Etapas do Processo},
    symbolic x coords={Prep. Material,Moldagem,Montagem,Integração,Testes,Esterilização,Embalagem},
    xtick=data,
    nodes near coords,
    nodes near coords align={vertical},
    width=\textwidth,
    height=8cm,
    ybar,
    bar width=20pt,
    x tick label style={rotate=45,anchor=east}
]
\addplot[fill=chart1] coordinates {
    (Prep. Material,300)
    (Moldagem,180)
    (Montagem,240)
    (Integração,300)
    (Testes,360)
    (Esterilização,1800)
    (Embalagem,120)
};
\end{axis}
\end{tikzpicture}
\caption{Duração das Etapas do Processo}
\label{fig:process-duration}
\end{figure}

\subsection{Alocação de Recursos}
\begin{table}[H]
\centering
\caption{Requisitos de Recursos do Processo}
\label{tab:resource-req}
\begin{tabular}{@{}lcccc@{}}
\toprule
Etapa do Processo & Operadores & Equipamentos & Verificações de Qualidade & Tempo de Setup (min) \\
\midrule
Preparação de Material & 2 & 2 & 2 & 15 \\
Moldagem por Injeção & 1 & 3 & 3 & 30 \\
Montagem de Componentes & 3 & 4 & 4 & 20 \\
Integração Eletrônica & 2 & 2 & 5 & 25 \\
Testes & 2 & 6 & 6 & 15 \\
Esterilização & 1 & 1 & 2 & 45 \\
Embalagem & 2 & 2 & 3 & 20 \\
\bottomrule
\end{tabular}
\end{table}

\section{Análise de Rede}
\begin{figure}[H]
\centering
\begin{tikzpicture}[
    node distance = 2cm and 3cm,
    event/.style={circle, draw, thick, minimum size=1.2cm},
    critical/.style={circle, draw, thick, fill=red!20, minimum size=1.2cm},
    activity/.style={->, >=stealth, thick}
]
% Eventos (nós)
\node[event] (1) {1};
\node[critical, right=of 1] (2) {2};
\node[event, above right=of 2] (3) {3};
\node[critical, right=of 2] (4) {4};
\node[event, below right=of 2] (5) {5};
\node[critical, right=of 4] (6) {6};

% Atividades (setas)
\draw[activity] (1) -- node[above] {Prep. Material (5)} (2);
\draw[activity] (2) -- node[above] {Moldagem (3)} (3);
\draw[activity] (2) -- node[above] {Montagem (4)} (4);
\draw[activity] (2) -- node[below] {Testes (6)} (5);
\draw[activity] (3) -- node[above right] {Integração (5)} (4);
\draw[activity] (4) -- node[above] {Embalagem (2)} (6);
\draw[activity] (5) -- node[below right] {CQ (3)} (6);

\end{tikzpicture}
\caption{Diagrama de Rede do Processo com Caminho Crítico}
\label{fig:network-diagram}
\end{figure}

\section{Pontos de Controle do Processo}

\subsection{Parâmetros de Controle de Qualidade}
\begin{figure}[H]
\centering
\begin{tikzpicture}
\begin{axis}[
    title={Distribuição de Pontos de Controle de Qualidade},
    ylabel={Número de Verificações},
    xlabel={Estágio de Produção},
    symbolic x coords={Matérias-Primas,Componentes,Montagem,Eletrônica,Produto Final},
    xtick=data,
    nodes near coords,
    nodes near coords align={vertical},
    width=\textwidth,
    height=8cm,
    ybar,
    bar width=25pt,
    x tick label style={rotate=45,anchor=east}
]
\addplot[fill=chart2] coordinates {
    (Matérias-Primas,3)
    (Componentes,4)
    (Montagem,5)
    (Eletrônica,6)
    (Produto Final,4)
};
\end{axis}
\end{tikzpicture}
\caption{Distribuição do Controle de Qualidade}
\label{fig:qc-distribution}
\end{figure}

\subsection{Parâmetros Críticos de Controle}
\begin{table}[H]
\centering
\caption{Parâmetros Críticos do Processo}
\label{tab:critical-parameters}
\begin{tabular}{@{}llcc@{}}
\toprule
Parâmetro & Especificação & Tolerância & Frequência de Medição \\
\midrule
Pressão de Injeção & 800 PSI & ±20 PSI & Cada ciclo \\
Temperatura do Molde & 180°C & ±5°C & Contínua \\
Torque de Montagem & 2,5 Nm & ±0,1 Nm & 100\% inspeção \\
Taxa de Fluxo & 0,1 mL/hr & ±0,005 mL/hr & Cada unidade \\
Desempenho da Bateria & 168 hrs & +2/-0 hrs & Cada unidade \\
\bottomrule
\end{tabular}
\end{table}

\section{Controle Detalhado do Processo}

\subsection{Gráficos de Controle}
% Gráficos X-bar e R para parâmetros críticos
\begin{figure}[H]
\centering
\begin{tikzpicture}
\begin{axis}[
    title={Gráfico X-bar: Pressão de Injeção},
    ylabel={Pressão (PSI)},
    xlabel={Número da Amostra},
    width=\textwidth,
    height=6cm,
    grid=major,
    legend pos=north west
]
\addplot[blue, thick] coordinates {
    (1,802) (2,798) (3,805) (4,799) (5,801)
    (6,797) (7,803) (8,800) (9,802) (10,799)
};
\addplot[red, dashed] coordinates {(0,820) (10,820)}; % LSC
\addplot[red, dashed] coordinates {(0,780) (10,780)}; % LIC
\addplot[green!50!black] coordinates {(0,800) (10,800)}; % Linha central
\legend{Medições,Limites de Controle,Meta}
\end{axis}
\end{tikzpicture}

\begin{tikzpicture}
\begin{axis}[
    title={Gráfico R: Amplitude da Pressão de Injeção},
    ylabel={Amplitude (PSI)},
    xlabel={Número da Amostra},
    width=\textwidth,
    height=6cm,
    grid=major,
    legend pos=north west
]
\addplot[blue, thick] coordinates {
    (1,5) (2,7) (3,4) (4,6) (5,5)
    (6,8) (7,4) (8,6) (9,5) (10,7)
};
\addplot[red, dashed] coordinates {(0,15) (10,15)}; % LSC
\addplot[red, dashed] coordinates {(0,0) (10,0)}; % LIC
\legend{Amplitude,Limites de Controle}
\end{axis}
\end{tikzpicture}
\caption{Gráficos de Controle para Pressão de Injeção}
\label{fig:control-charts}
\end{figure}

\section{Análise da Capacidade do Processo}

\subsection{Análise de Estabilidade do Processo}
\begin{figure}[H]
\centering
\begin{tikzpicture}
\begin{axis}[
    title={Gráfico de Amplitude Móvel - Torque de Montagem},
    ylabel={Amplitude Móvel (Nm)},
    xlabel={Número da Amostra},
    width=\textwidth,
    height=6cm,
    grid=major,
    legend pos=north west
]
\addplot[blue, thick] coordinates {
    (1,0.05) (2,0.07) (3,0.04) (4,0.06) (5,0.05)
    (6,0.08) (7,0.04) (8,0.06) (9,0.05) (10,0.07)
    (11,0.06) (12,0.05) (13,0.07) (14,0.04) (15,0.06)
};
\addplot[red, dashed] coordinates {(0,0.15) (15,0.15)}; % LSC
\addplot[green!50!black] coordinates {(0,0.06) (15,0.06)}; % Linha Central
\legend{Amplitude Móvel,LSC,Média}
\end{axis}
\end{tikzpicture}

\begin{tikzpicture}
\begin{axis}[
    title={Gráfico CUSUM - Torque de Montagem},
    ylabel={Soma Cumulativa},
    xlabel={Número da Amostra},
    width=\textwidth,
    height=6cm,
    grid=major,
    legend pos=north west
]
\addplot[blue, thick] coordinates {
    (1,0.02) (2,0.05) (3,0.03) (4,0.06) (5,0.04)
    (6,0.07) (7,0.05) (8,0.04) (9,0.06) (10,0.03)
    (11,0.05) (12,0.04) (13,0.06) (14,0.03) (15,0.05)
};
\addplot[red, dashed] coordinates {(0,0.1) (15,0.1)}; % Limite superior de decisão
\addplot[red, dashed] coordinates {(0,-0.1) (15,-0.1)}; % Limite inferior de decisão
\legend{CUSUM,Limites de Decisão}
\end{axis}
\end{tikzpicture}
\caption{Gráficos de Controle Estatístico do Processo}
\label{fig:spc-charts}
\end{figure}

\subsection{Capacidade do Processo - Análise Detalhada}
\begin{figure}[H]
\centering
\begin{tikzpicture}
\begin{axis}[
    title={Análise de Capacidade do Processo - Taxa de Fluxo},
    xlabel={Taxa de Fluxo (mL/hr)},
    ylabel={Frequência},
    width=\textwidth,
    height=8cm,
    grid=major,
    legend pos=north west
]
% Curva de distribuição normal
\addplot[smooth, thick, domain=0.09:0.11] {1000*exp(-(x-0.1)^2/0.0001)};
% Limites de especificação
\addplot[red, dashed] coordinates {(0.095,0) (0.095,1000)};
\addplot[red, dashed] coordinates {(0.105,0) (0.105,1000)};
% Linha alvo
\addplot[green!50!black] coordinates {(0.1,0) (0.1,1000)};
\legend{Distribuição,Limites de Especificação,Meta}

% Adicionar valores de Cp e Cpk
\node[anchor=north west] at (axis cs:0.091,900) 
    {$C_p = 1,67$\\$C_{pk} = 1,65$};
\end{axis}
\end{tikzpicture}
\caption{Análise de Capacidade do Processo}
\label{fig:process-capability}
\end{figure}

\subsection{Métricas de Desempenho do Processo}
\begin{table}[H]
\centering
\caption{Resumo do Desempenho do Processo}
\label{tab:process-performance}
\begin{tabular}{@{}lcccc@{}}
\toprule
Parâmetro & Cp & Cpk & Pp & Ppk \\
\midrule
Pressão de Injeção & 1,67 & 1,65 & 1,63 & 1,60 \\
Torque de Montagem & 1,55 & 1,52 & 1,50 & 1,48 \\
Taxa de Fluxo & 1,70 & 1,68 & 1,65 & 1,62 \\
Desempenho da Bateria & 1,45 & 1,42 & 1,40 & 1,38 \\
\bottomrule
\end{tabular}
\end{table}

\subsection{Análise de Tendências do Processo}
\begin{figure}[H]
\centering
\begin{tikzpicture}
\begin{axis}[
    ybar=5pt,
    title={Tendências de Capacidade por Linha de Produto},
    ylabel={Índice de Capacidade},
    symbolic x coords={X1,X1 Pro,Mini},
    xtick=data,
    nodes near coords,
    width=\textwidth,
    height=8cm,
    legend entries={Cp,Cpk},
    legend pos=north west,
    bar width=15pt
]
\addplot[fill=chart1] coordinates {
    (X1,1.67)
    (X1 Pro,1.72)
    (Mini,1.65)
};
\addplot[fill=chart2] coordinates {
    (X1,1.65)
    (X1 Pro,1.70)
    (Mini,1.62)
};
\end{axis}
\end{tikzpicture}
\caption{Índices de Capacidade por Produto}
\label{fig:capability-indices}
\end{figure}

\subsection{Distribuição de Performance}
\begin{figure}[H]
\centering
\begin{tikzpicture}
\begin{axis}[
    scatter/classes={
        a={mark=o,draw=black,fill=chart1},
        b={mark=square,draw=black,fill=chart2},
        c={mark=triangle,draw=black,fill=chart3}
    },
    title={Distribuição de Desempenho do Processo},
    xlabel={Amostras},
    ylabel={Valor Medido},
    width=\textwidth,
    height=8cm,
    grid=major,
    legend entries={X1,X1 Pro,Mini}
]
\addplot[scatter,only marks,scatter src=explicit symbolic]
coordinates {
    (1,98) [a] (2,99) [a] (3,97) [a] (4,98.5) [a] (5,99) [a]
    (1,99) [b] (2,98.5) [b] (3,99.5) [b] (4,98) [b] (5,99) [b]
    (1,97) [c] (2,98) [c] (3,97.5) [c] (4,98) [c] (5,97.5) [c]
};
\end{axis}
\end{tikzpicture}
\caption{Distribuição de Desempenho do Processo}
\label{fig:performance-dist}
\end{figure}

\section{Cronograma de Produção}
% Gráfico de Gantt para cronograma de produção
\begin{figure}[H]
\centering
\begin{ganttchart}[
    vgrid,
    hgrid,
    bar/.style={fill=chart1},
    bar height=.6,
    group right shift=0,
    group height=.3,
    group top shift=.6,
    today=10
    ]{1}{30}
    \gantttitle{Cronograma de Produção (Dias)}{30} \\
    \gantttitlelist{1,...,30}{1} \\
    \ganttgroup{Lote A}{1}{10} \\
    \ganttbar{Preparação de Material}{1}{2} \\
    \ganttbar{Moldagem}{3}{5} \\
    \ganttbar{Montagem}{6}{8} \\
    \ganttbar{Testes}{9}{10} \\
    \ganttgroup{Lote B}{11}{20} \\
    \ganttbar{Preparação de Material}{11}{12} \\
    \ganttbar{Moldagem}{13}{15} \\
    \ganttbar{Montagem}{16}{18} \\
    \ganttbar{Testes}{19}{20} \\
    \ganttgroup{Lote C}{21}{30} \\
    \ganttbar{Preparação de Material}{21}{22} \\
    \ganttbar{Moldagem}{23}{25} \\
    \ganttbar{Montagem}{26}{28} \\
    \ganttbar{Testes}{29}{30}
\end{ganttchart}
\caption{Gráfico de Gantt do Cronograma de Produção}
\label{fig:gantt-chart}
\end{figure}

\section{Análise de Controle Estatístico do Processo}
\subsection{Análise de Estabilidade do Processo}
\begin{figure}[H]
\centering
\begin{tikzpicture}
\begin{axis}[
    title={Gráfico de Amplitude Móvel - Torque de Montagem},
    ylabel={Amplitude Móvel (Nm)},
    xlabel={Número da Amostra},
    width=\textwidth,
    height=6cm,
    grid=major,
    legend pos=north west
]
\addplot[blue, thick] coordinates {
    (1,0.05) (2,0.07) (3,0.04) (4,0.06) (5,0.05)
    (6,0.08) (7,0.04) (8,0.06) (9,0.05) (10,0.07)
    (11,0.06) (12,0.05) (13,0.07) (14,0.04) (15,0.06)
};
\addplot[red, dashed] coordinates {(0,0.15) (15,0.15)}; % LSC
\addplot[green!50!black] coordinates {(0,0.06) (15,0.06)}; % Linha central
\legend{Amplitude Móvel,LSC,Média}
\end{axis}
\end{tikzpicture}
\begin{tikzpicture}
\begin{axis}[
    title={Gráfico CUSUM - Torque de Montagem},
    ylabel={Soma Cumulativa},
    xlabel={Número da Amostra},
    width=\textwidth,
    height=6cm,
    grid=major,
    legend pos=north west
]
\addplot[blue, thick] coordinates {
    (1,0.02) (2,0.05) (3,0.03) (4,0.06) (5,0.04)
    (6,0.07) (7,0.05) (8,0.04) (9,0.06) (10,0.03)
    (11,0.05) (12,0.04) (13,0.06) (14,0.03) (15,0.05)
};
\addplot[red, dashed] coordinates {(0,0.1) (15,0.1)}; % Limite superior de decisão
\addplot[red, dashed] coordinates {(0,-0.1) (15,-0.1)}; % Limite inferior de decisão
\legend{CUSUM,Limites de Decisão}
\end{axis}
\end{tikzpicture}
\caption{Gráficos de Controle Estatístico do Processo}
\label{fig:spc-charts}
\end{figure}

\subsection{Métricas de Desempenho do Processo}
\begin{table}[H]
\centering
\caption{Resumo do Desempenho do Processo}
\label{tab:process-performance}
\begin{tabular}{@{}lcccc@{}}
\toprule
Parâmetro & Cp & Cpk & Pp & Ppk \\
\midrule
Pressão de Injeção & 1.67 & 1.65 & 1.63 & 1.60 \\
Torque de Montagem & 1.55 & 1.52 & 1.50 & 1.48 \\
Taxa de Fluxo & 1.70 & 1.68 & 1.65 & 1.62 \\
Desempenho da Bateria & 1.45 & 1.42 & 1.40 & 1.38 \\
\bottomrule
\end{tabular}
\end{table}

\chapter{Análise dos dados}

\section{Análise de Controle de Qualidade}
\subsection{Parâmetros Críticos de Qualidade}
\begin{figure}[H]
\centering
\begin{tikzpicture}
\begin{axis}[
    ybar,
    title={Gráfico de Pareto de Defeitos de Qualidade},
    ylabel={Número de Ocorrências},
    ylabel near ticks,
    symbolic x coords={
        Montagem,
        Moldagem,
        Eletrônica,
        Calibração,
        Embalagem,
        Outros
    },
    xtick=data,
    nodes near coords,
    x tick label style={rotate=45,anchor=east},
    width=\textwidth,
    height=8cm,
    bar width=25pt
]
\addplot[fill=chart1] coordinates {
    (Montagem,45)
    (Moldagem,38)
    (Eletrônica,25)
    (Calibração,18)
    (Embalagem,12)
    (Outros,8)
};
\end{axis}
% Adicionar linha de porcentagem cumulativa
\begin{axis}[
    axis y line*=right,
    ylabel={Porcentagem Cumulativa},
    ylabel near ticks,
    width=\textwidth,
    height=8cm,
    symbolic x coords={
        Montagem,
        Moldagem,
        Eletrônica,
        Calibração,
        Embalagem,
        Outros
    },
    xtick=data,
    x tick label style={rotate=45,anchor=east}
]
\addplot[smooth,red,thick,mark=*] coordinates {
    (Montagem,30.8)
    (Moldagem,56.8)
    (Eletrônica,74.0)
    (Calibração,86.3)
    (Embalagem,94.5)
    (Outros,100)
};
\end{axis}
\end{tikzpicture}
\caption{Análise de Pareto dos Defeitos de Qualidade}
\label{fig:quality-pareto}
\end{figure}
\subsection{Limites de Controle do Processo}
\begin{table}[H]
\centering
\caption{Limites de Controle Estabelecidos por Processo}
\label{tab:control-limits}
\begin{tabular}{@{}lcccc@{}}
\toprule
Parâmetro do Processo & Alvo & LIE & LSE & Crítico para Qualidade \\
\midrule
Pressão de Injeção (PSI) & 800 & 780 & 820 & Sim \\
Torque de Montagem (Nm) & 2.5 & 2.4 & 2.6 & Sim \\
Taxa de Fluxo (mL/hr) & 0.1 & 0.095 & 0.105 & Sim \\
Vida da Bateria (hrs) & 168 & 168 & 170 & Sim \\
\bottomrule
\end{tabular}
\end{table}

\section{Análise de Capacidade do Processo}
\subsection{Índices de Capacidade por Linha de Produto}
\begin{figure}[H]
\centering
\begin{tikzpicture}
\begin{axis}[
   ybar,
   title={Comparação de Capacidade do Processo},
   ylabel={Índice de Capacidade},
   symbolic x coords={X1,X1 Pro,Mini},
   xtick=data,
   nodes near coords,
   width=\textwidth,
   height=8cm,
   legend entries={Cp,Cpk},
   legend pos=north west,
   bar width=15pt
]
\addplot[fill=chart1] coordinates {
   (X1,1.67)
   (X1 Pro,1.72)
   (Mini,1.65)
};
\addplot[fill=chart2] coordinates {
   (X1,1.65)
   (X1 Pro,1.70)
   (Mini,1.62)
};
\end{axis}
\end{tikzpicture}
\caption{Índices de Capacidade por Produto}
\label{fig:capability-indices}
\end{figure}
\subsection{Análise de Desempenho do Processo}
\begin{figure}[H]
\centering
\begin{tikzpicture}
\begin{axis}[
   scatter/classes={
       a={mark=o,draw=black,fill=chart1},
       b={mark=square,draw=black,fill=chart2},
       c={mark=triangle,draw=black,fill=chart3}
   },
   title={Distribuição de Desempenho do Processo},
   xlabel={Amostras},
   ylabel={Valor Medido},
   width=\textwidth,
   height=8cm,
   grid=major,
   legend entries={X1,X1 Pro,Mini}
]
\addplot[scatter,only marks,scatter src=explicit symbolic]
coordinates {
   (1,98) [a] (2,99) [a] (3,97) [a] (4,98.5) [a] (5,99) [a]
   (1,99) [b] (2,98.5) [b] (3,99.5) [b] (4,98) [b] (5,99) [b]
   (1,97) [c] (2,98) [c] (3,97.5) [c] (4,98) [c] (5,97.5) [c]
};
\end{axis}
\end{tikzpicture}
\caption{Distribuição de Desempenho do Processo}
\label{fig:performance-dist}
\end{figure}

\section{Análise Estatística}
\subsection{Análise de Correlação}
\begin{figure}[H]
\centering
\begin{tikzpicture}
\begin{axis}[
   title={Matriz de Correlação de Parâmetros},
   xlabel={Parâmetros do Processo},
   ylabel={Métricas de Qualidade},
   width=\textwidth,
   height=8cm,
   enlargelimits=0.15,
   colorbar,
   colormap={correlation}{
       color(0)=(white)
       color(1)=(chart1)
   }
]
\addplot[matrix plot*,point meta=explicit] table[meta=c] {
   x y c
   0 0 1.00
   0 1 0.85
   0 2 0.65
   0 3 0.45
   1 0 0.85
   1 1 1.00
   1 2 0.75
   1 3 0.55
   2 0 0.65
   2 1 0.75
   2 2 1.00
   2 3 0.70
   3 0 0.45
   3 1 0.55
   3 2 0.70
   3 3 1.00
};
\end{axis}
\end{tikzpicture}
\caption{Análise de Correlação de Parâmetros}
\label{fig:correlation-matrix}
\end{figure}

\section{Recomendações de Melhoria do Processo}
\begin{table}[H]
\centering
\caption{Oportunidades de Melhoria do Processo}
\label{tab:improvements}
\begin{tabular}{@{}llc@{}}
\toprule
Área do Processo & Ação de Melhoria & Impacto Esperado (\%) \\
\midrule
Montagem & Controle automatizado de torque & 15 \\
Moldagem & Otimização de temperatura & 12 \\
Testes & Calibração aprimorada & 8 \\
Embalagem & Inspeção automatizada & 5 \\
\bottomrule
\end{tabular}
\end{table}
\subsection{Análise de Custo-Benefício}
\begin{figure}[H]
\centering
\begin{tikzpicture}
\begin{axis}[
   ybar,
   title={Análise de Investimento vs. Retorno},
   ylabel={Custo/Benefício (k\$)},
   symbolic x coords={Montagem,Moldagem,Testes,Embalagem},
   xtick=data,
   nodes near coords,
   x tick label style={rotate=45,anchor=east},
   width=\textwidth,
   height=8cm,
   legend entries={Investimento,Retorno Anual},
   legend pos=north west,
   bar width=15pt
]
\addplot[fill=chart1] coordinates {
   (Montagem,150)
   (Moldagem,120)
   (Testes,80)
   (Embalagem,50)
};
\addplot[fill=chart2] coordinates {
   (Montagem,225)
   (Moldagem,168)
   (Testes,104)
   (Embalagem,60)
};
\end{axis}
\end{tikzpicture}
\caption{Análise de Investimento e Retorno}
\label{fig:cost-benefit}
\end{figure}


\chapter{Melhorias do processo}

\section{Proposta de Automação}
\subsection{Comparação do Processo Atual vs. Proposto}
\begin{figure}[H]
\centering
\begin{tikzpicture}
\begin{axis}[
   ybar=5pt,
   title={Comparação de Tempo do Processo: Manual vs. Automatizado},
   ylabel={Tempo (minutos)},
   symbolic x coords={Prep. Material,Montagem,Testes,Embalagem},
   xtick=data,
   nodes near coords,
   x tick label style={rotate=45,anchor=east},
   width=\textwidth,
   height=8cm,
   legend entries={Processo Atual,Processo Automatizado},
   legend pos=north west,
   bar width=15pt
]
\addplot[fill=chart1] coordinates {
   (Prep. Material,30)
   (Montagem,45)
   (Testes,35)
   (Embalagem,25)
};
\addplot[fill=chart2] coordinates {
   (Prep. Material,15)
   (Montagem,20)
   (Testes,15)
   (Embalagem,10)
};
\end{axis}
\end{tikzpicture}
\caption{Comparação de Tempo do Processo}
\label{fig:time-comparison}
\end{figure}
\subsection{Análise de Investimento em Automação}
\begin{table}[H]
\centering
\caption{Detalhes do Investimento em Automação}
\label{tab:automation-investment}
\begin{tabular}{@{}llrr@{}}
\toprule
Área do Processo & Equipamento & Investimento (k\$) & ROI (\%) \\
\midrule
Manuseio de Material & Sistema Robótico & 250 & 45 \\
Montagem & Linha Automatizada & 450 & 65 \\
Testes & Sistema de Visão & 180 & 55 \\
Embalagem & Esteira Inteligente & 120 & 40 \\
\bottomrule
\end{tabular}
\end{table}

\section{Estratégia de Implementação}
\subsection{Cronograma de Implementação}
\begin{figure}[H]
\centering
\begin{ganttchart}[
   vgrid,
   hgrid,
   bar/.style={fill=chart1},
   bar height=.6,
   group right shift=0,
   group height=.3,
   group top shift=.6
   ]{1}{24}
   \gantttitle{Cronograma de Implementação (Meses)}{24} \\
   \gantttitlelist{1,...,24}{1} \\
   \ganttgroup{Fase 1: Manuseio de Material}{1}{8} \\
   \ganttbar{Instalação de Equipamentos}{1}{3} \\
   \ganttbar{Testes \& Validação}{4}{6} \\
   \ganttbar{Treinamento da Equipe}{6}{8} \\
   \ganttgroup{Fase 2: Montagem}{7}{16} \\
   \ganttbar{Configuração da Linha}{7}{10} \\
   \ganttbar{Integração}{11}{13} \\
   \ganttbar{Otimização}{14}{16} \\
   \ganttgroup{Fase 3: Testes \& Embalagem}{15}{24} \\
   \ganttbar{Instalação do Sistema de Visão}{15}{18} \\
   \ganttbar{Configuração da Esteira}{19}{21} \\
   \ganttbar{Integração Final}{22}{24}
\end{ganttchart}
\caption{Cronograma de Implementação}
\label{fig:implementation-timeline}
\end{figure}
\subsection{Análise de Riscos}
\begin{figure}[H]
\centering
\begin{tikzpicture}
\begin{axis}[
   scatter/classes={
       high={mark=*,draw=red,fill=red},
       medium={mark=square*,draw=orange,fill=orange},
       low={mark=triangle*,draw=green,fill=green}
   },
   title={Matriz de Avaliação de Riscos},
   xlabel={Impacto},
   ylabel={Probabilidade},
   width=\textwidth,
   height=8cm,
   grid=major,
   xtick={1,2,3,4,5},
   ytick={1,2,3,4,5},
   legend entries={Risco Alto,Risco Médio,Risco Baixo}
]
% Plotar pontos de risco
\addplot[scatter,only marks,scatter src=explicit symbolic]
coordinates {
   (4,4) [high]    % Interrupção da Produção
   (4,3) [high]    % Problemas de Integração
   (3,3) [medium]  % Treinamento da Equipe
   (3,2) [medium]  % Atrasos de Equipamentos
   (2,2) [low]     % Estouro de Custo
   (2,1) [low]     % Documentação
};
\end{axis}
\end{tikzpicture}
\caption{Matriz de Avaliação de Riscos}
\label{fig:risk-matrix}
\end{figure}

\section{Benefícios Esperados}
\subsection{Melhorias de Produtividade}
\begin{figure}[H]
\centering
\begin{tikzpicture}
\begin{axis}[
   title={Aumento Projetado de Produtividade},
   xlabel={Meses Após Implementação},
   ylabel={Produtividade (\%)},
   width=\textwidth,
   height=8cm,
   grid=major,
   legend pos=south east
]
\addplot[smooth,thick,chart1] coordinates {
   (0,100)
   (3,110)
   (6,125)
   (12,140)
   (18,150)
   (24,160)
};
\addplot[dashed,thick,chart2] coordinates {
   (0,100)
   (24,100)
};
\legend{Processo Automatizado,Processo Atual}
\end{axis}
\end{tikzpicture}
\caption{Projeção de Produtividade}
\label{fig:productivity-projection}
\end{figure}
\subsection{Melhorias de Qualidade}
\begin{table}[H]
\centering
\caption{Melhorias de Qualidade Esperadas}
\label{tab:quality-improvements}
\begin{tabular}{@{}lrr@{}}
\toprule
Métrica de Qualidade & Atual & Esperado \\
\midrule
Rendimento de Primeira Passagem (\%) & 95.5 & 98.5 \\
Taxa de Defeitos (PPM) & 2500 & 500 \\
Taxa de Refugo (\%) & 3.2 & 1.1 \\
Devoluções de Clientes (\%) & 0.5 & 0.1 \\
\bottomrule
\end{tabular}
\end{table}

\section{Análise de Custos}
\subsection{Retorno sobre Investimento}
\begin{figure}[H]
\centering
\begin{tikzpicture}
\begin{axis}[
   title={Projeção de ROI Cumulativo},
   xlabel={Ano},
   ylabel={Retorno (\%)},
   width=\textwidth,
   height=8cm,
   grid=major,
   legend pos=north west
]
\addplot[smooth,thick,chart1] coordinates {
   (0,0)
   (1,15)
   (2,45)
   (3,85)
   (4,135)
   (5,195)
};
\addplot[dashed,red] coordinates {
   (0,0)
   (5,100)
};
\legend{Retorno Projetado,Ponto de Equilíbrio}
\end{axis}
\end{tikzpicture}
\caption{Projeção de ROI}
\label{fig:roi-projection}
\end{figure}
\subsection{Redução de Custos Operacionais}
\begin{figure}[H]
\centering
\begin{tikzpicture}
\begin{axis}[
   ybar,
   title={Redução Anual de Custos Operacionais},
   ylabel={Redução de Custos (\%)},
   symbolic x coords={Mão de Obra,Manutenção,Qualidade,Energia,Materiais},
   xtick=data,
   nodes near coords,
   x tick label style={rotate=45,anchor=east},
   width=\textwidth,
   height=8cm,
   bar width=25pt
]
\addplot[fill=chart1] coordinates {
   (Mão de Obra,45)
   (Manutenção,25)
   (Qualidade,35)
   (Energia,15)
   (Materiais,20)
};
\end{axis}
\end{tikzpicture}
\caption{Redução de Custos Operacionais por Categoria}
\label{fig:cost-reduction}
\end{figure}

\chapter{Conclusão e recomendações}

\section{Resumo do Projeto}
\subsection{Principais Conquistas}
\begin{figure}[H]
\centering
\begin{tikzpicture}
\begin{axis}[
   ybar,
   title={Alcance dos Objetivos do Projeto},
   ylabel={Taxa de Alcance (\%)},
   symbolic x coords={Qualidade,Produtividade,Automação,Redução de Custos},
   xtick=data,
   nodes near coords,
   x tick label style={rotate=45,anchor=east},
   width=\textwidth,
   height=8cm,
   bar width=25pt
]
\addplot[fill=chart1] coordinates {
   (Qualidade,98)
   (Produtividade,95)
   (Automação,92)
   (Redução de Custos,88)
};
\end{axis}
\end{tikzpicture}
\caption{Alcance dos Objetivos do Projeto}
\label{fig:objectives-achievement}
\end{figure}
\subsection{Impacto Financeiro}
\begin{table}[H]
\centering
\caption{Resumo do Impacto Financeiro}
\label{tab:financial-summary}
\begin{tabular}{@{}lrr@{}}
\toprule
Métrica & Meta & Alcançado \\
\midrule
Retorno do Investimento (\%) & 150 & 165 \\
Redução de Custos (\%) & 25 & 28 \\
Aumento de Produtividade (\%) & 40 & 45 \\
Melhoria da Qualidade (\%) & 3 & 3.5 \\
\bottomrule
\end{tabular}
\end{table}

\section{Recomendações Finais}
\begin{enumerate}
   \item Continuar o investimento em automação e transformação digital
   \item Aprimorar programas de treinamento e desenvolvimento dos funcionários
   \item Fortalecer relacionamentos com fornecedores e programas de qualidade
   \item Implementar análise avançada e manutenção preditiva
   \item Desenvolver sistema abrangente de documentação
\end{enumerate}
% Tabela final de resumo
\begin{table}[H]
\centering
\caption{Métricas de Sucesso do Projeto}
\label{tab:success-metrics}
\begin{tabular}{@{}lcc@{}}
\toprule
Métrica de Sucesso & Meta & Alcançado \\
\midrule
Conclusão do Projeto & 100\% & 98\% \\
Aderência ao Orçamento & ±5\% & +3\% \\
Conformidade com Cronograma & 100\% & 95\% \\
Objetivos de Qualidade & 100\% & 98\% \\
\bottomrule
\end{tabular}
\end{table}

\end{document}